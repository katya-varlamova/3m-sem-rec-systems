\documentclass[12pt]{report}
\usepackage[utf8]{inputenc}
\usepackage[russian]{babel}
%\usepackage[14pt]{extsizes}
\usepackage{listings}
\usepackage{graphicx}
\usepackage{amsmath,amsfonts,amssymb,amsthm,mathtools} 
\usepackage{pgfplots}
\usepackage{filecontents}
\usepackage{indentfirst}
\usepackage{eucal}
\usepackage{amsmath}
\usepackage{enumitem}
\frenchspacing

\usepackage{indentfirst} % Красная строка


%\usetikzlibrary{datavisualization}
%\usetikzlibrary{datavisualization.formats.functions}

\usepackage{amsmath}




% Для листинга кода:
\lstset{ %
language=haskell,                 % выбор языка для подсветки (здесь это С)
basicstyle=\small\sffamily, % размер и начертание шрифта для подсветки кода
numbers=left,               % где поставить нумерацию строк (слева\справа)
numberstyle=\tiny,           % размер шрифта для номеров строк
stepnumber=1,                   % размер шага между двумя номерами строк
numbersep=5pt,                % как далеко отстоят номера строк от подсвечиваемого кода
showspaces=false,            % показывать или нет пробелы специальными отступами
showstringspaces=false,      % показывать или нет пробелы в строках
showtabs=false,             % показывать или нет табуляцию в строках
frame=single,              % рисовать рамку вокруг кода
tabsize=2,                 % размер табуляции по умолчанию равен 2 пробелам
captionpos=t,              % позиция заголовка вверху [t] или внизу [b] 
breaklines=true,           % автоматически переносить строки (да\нет)
breakatwhitespace=false, % переносить строки только если есть пробел
escapeinside={\#*}{*)}   % если нужно добавить комментарии в коде
}

\usepackage[left=2cm,right=2cm, top=2cm,bottom=2cm,bindingoffset=0cm]{geometry}
% Для измененных титулов глав:
\usepackage{titlesec, blindtext, color} % подключаем нужные пакеты
\definecolor{gray75}{gray}{0.75} % определяем цвет
\newcommand{\hsp}{\hspace{20pt}} % длина линии в 20pt
% titleformat определяет стиль
\titleformat{\chapter}[hang]{\Huge\bfseries}{\thechapter\hsp\textcolor{gray75}{|}\hsp}{0pt}{\Huge\bfseries}


% plot
\usepackage{pgfplots}
\usepackage{filecontents}
\usetikzlibrary{datavisualization}
\usetikzlibrary{datavisualization.formats.functions}
\RequirePackage[
  style=gost-numeric,
  language=auto,
  autolang=other,
  sorting=none,
]{biblatex}

\addbibresource{bib.bib}
\begin{document}
%\def\chaptername{} % убирает "Глава"
\thispagestyle{empty}
\begin{titlepage}
	\noindent \begin{minipage}{0.15\textwidth}
	\includegraphics[width=\linewidth]{b_logo}
	\end{minipage}
	\noindent\begin{minipage}{0.9\textwidth}\centering
		\textbf{Министерство науки и высшего образования Российской Федерации}\\
		\textbf{Федеральное государственное бюджетное образовательное учреждение высшего образования}\\
		\textbf{~~~«Московский государственный технический университет имени Н.Э.~Баумана}\\
		\textbf{(национальный исследовательский университет)»}\\
		\textbf{(МГТУ им. Н.Э.~Баумана)}
	\end{minipage}
	
	\noindent\rule{18cm}{3pt}
	\newline\newline
	\noindent ФАКУЛЬТЕТ $\underline{\text{«Информатика и системы управления»}}$ \newline\newline
	\noindent КАФЕДРА $\underline{\text{«Программное обеспечение ЭВМ и информационные технологии»}}$\newline\newline\newline\newline\newline
	
	
	\begin{center}
		\noindent\begin{minipage}{1.3\textwidth}\centering
			\Large\textbf{  Отчёт по лабораторной работе №4 по дисциплине}\newline
			\textbf{ "Проектирование рекомендательных систем"}\newline\newline
		\end{minipage}
	\end{center}
	
	\noindent\textbf{Тема} $\underline{\text{Алгоритмы построения рекомендаций с использованием матричной факторизации}}$\newline\newline
	\noindent\textbf{Студент} $\underline{\text{Варламова Е. А.}}$\newline\newline
	\noindent\textbf{Группа} $\underline{\text{ИУ7-33М}}$\newline\newline
	\noindent\textbf{Оценка (баллы)} $\underline{\text{~~~~~~~~~~~~~~~~~~~~~~~~~~~}}$\newline\newline
	\noindent\textbf{Преподаватели} $\underline{\text{Быстрицкая А.Ю.}}$\newline\newline\newline
	
	\begin{center}
		\vfill
		Москва~---~\the\year
		~г.
	\end{center}
\end{titlepage}
\large
\setcounter{page}{2}
\def\contentsname{СОДЕРЖАНИЕ}
\renewcommand{\contentsname}{СОДЕРЖАНИЕ}
\tableofcontents
\renewcommand\labelitemi{---}
\newpage

\chapter*{ВВЕДЕНИЕ}
\addcontentsline{toc}{section}{Введение}

Цель работы -- изучить алгоритмы матричной факторизации на примере SVD применительно к алгоритму коллаборативной фильтрации для построения рекомендаций.

Для достижения поставленной цели потребуется:
\begin{itemize}
	\item привести описание алгоритма коллаборативной фильтрации по пользователю;
    \item привести описание алгоритма матричной факторизации;
	\item привести описание используемых для исследования данных;
	\item провести исследование скорости работы алгоритма коллаборативной фильтрации и ошибок рекомендаций в зависимости параметров матричной факторизации, сравнить с алгоритмом без матричной факторизации.
\end{itemize}

\pagebreak

\pagebreak


\chapter{Аналитический раздел}

\section{Коллаборативная фильтрация}
Коллаборативная фильтрация – это метод рекомендации, при котором анализируется только реакция пользователей на объекты – оценки, которые выставляют пользователи.
Оценки могут быть как явными, так и неявными (например, длитель- ность нахождения пользователя на странице товара). Целью фильтрации яв- ляется предсказание оценки пользователем по оценкам других. Чем больше имеется оценок, тем точнее получатся рекомендации. 
\subsection{Коллаборативная фильтрация по пользователю}

В данном методе предполагается, что пользователи, которые в прошлом имели похожие предпочтения, будут иметь похожие предпочтения и в буду- щем. Для построения рекомендаций с использованием этого метода выпол- няются следующие шаги:
\begin{enumerate}
    \item Создание матрицы, где строки представляют пользователей, а столбцы объекты; значения в ячейках матрицы отражают оценки или действия пользователей по отношению к объектам;
    \item Вычисление сходства между пользователями (обычно используются ко- синусное сходство или корреляция Пирсона);
    \item Для конкретного пользователя генерируются рекомендации на основе сходства с другими пользователями через агрегацию оценок или дей- ствий похожих пользователей.
\end{enumerate}

\section{Матричная факторизация}
Матричная факторизация -- это класс алгоритмов коллаборативной фильтрации, используемых в рекомендательных системах. Данные алгоритмы работают путем разложения матрицы взаимодействия пользователя с объектами на произведение двух прямоугольных матриц меньшей размерности. Зачастую матричная факторизация используется для улучшения качества персонализированных рекомендаций, позволяя выявить скрытые паттерны и взаимосвязи между пользователями и товарами.\cite{factorization}

Методы матричной факторизации в рекомендательных системах обладают следующими аспектами:

\begin{itemize}
	\item Снижение размерности -- уменьшение объема вычислений и увеличение эффективности;
	\item Скрытые факторы -- данные методы предполагают, что в системе присутствуют некоторые латентные признаки, которые влияют на предпочтения пользователей и характеристики товаров;
	\item Эффективность работы с разреженными данными -- матричная факторизация может эффективно работать с разреженными данными, заполняя недостающие значения.
\end{itemize}

\pagebreak

\chapter{Конструкторский раздел}

\section{MovieLens Rating}
В качестве источника данных был взят датасет, располагающийся в свободном доступе на веб-сайте MovieLens [2]. Набор данных включает в себя множество записей, поля которых описывают идентификатор пользова- теля, объект, оценку (от 1 до 5) и время появления рецензии. Предобработки проводить не потребовалось.


\pagebreak

\chapter{Технологический раздел}

\section{Средства реализации}

В качестве используемого был выбран язык программирования Python \cite{Python}.

Данный выбор обусловлен следующими факторами:
\begin{itemize}
	\item Большое количество исчерпывающей документации;
	\item Широкий выбор доступных библиотек для разработки;
	\item Простота синтаксиса языка и высокая скорость разработки.
\end{itemize} 

При написании программного продукта использовалась среда разработки Visual Studio Code. Данный выбор обусловлен тем, что данная среда распространяется по свободной лицензии, поставляется для конечного пользователя с открытым исходным кодом, а также имеет большое число расширений, ускоряющих разработку.

\section{Библиотеки}

При анализе и обработке датасета, а также для решения поставленных задач использовались библиотеки:
\begin{itemize}
	\item pandas;
	\item numpy;
	\item matplotlib \cite{matplotlib};
	\item sklearn \cite{sklearn};
    \item LibRecommender \cite{libreco}.
\end{itemize}

Данные библиотеки позволили полностью покрыть спектр потребностей при выполнении работы.


\chapter{Исследовательский раздел}
\section{Условия исследований}
Исследование проводилось на персональном вычислительной машине со следующими характеристиками:

\begin{itemize}
\item процессор Intel Core i5,
\item операционная система MacOS Big Sur
\item 8 Гб оперативной памяти.
\end{itemize}

Временные затраты определялись с использованием библиотеки time.

На рисунке \ref{img:time4} представлено исследование алгоритма коллаборативной фильтрации с использованием матричной факторизации в зависмости от количества факторов по времени работы и ошибкам рекомендаций. Кроме того, для сравнения добавлен алгоритм из библиотеки \cite{libreco}, не использующий матричную факторизацию.

\begin{figure}[H]
	\centering
	\includegraphics[width=\textwidth]{inc/res.png}
	\caption{Сравнение алгоритмов}
	\label{img:time4}
\end{figure}


Видно, что алгоритм, использующий матричную факторизацию более точен и работает быстрее, чем алгоритм без нее.


\chapter*{ЗАКЛЮЧЕНИЕ}
\addcontentsline{toc}{section}{ЗАКЛЮЧЕНИЕ}
В ходе выполнения рбаоты были изучены алгоритмы матричной факторизации на примере SVD применительно к алгоритму коллаборативной фильтрации для построения рекомендаций.

Для достижения поставленной цели потребовалось:
\begin{itemize}
	\item привести описание алгоритма коллаборативной фильтрации по пользователю;
    \item привести описание алгоритма матричной факторизации;
	\item привести описание используемых для исследования данных;
	\item провести исследование скорости работы алгоритма коллаборативной фильтрации и ошибок рекомендаций в зависимости параметров матричной факторизации, сравнить с алгоритмом без матричной факторизации.
\end{itemize}

Проведенные исследования показали, что алгоритм, использующий матричную факторизацию более точен и работает быстрее, чем алгоритм без нее.

\pagebreak

\printbibliography[title={СПИСОК ИСПОЛЬЗОВАННЫХ\\ ИСТОЧНИКОВ}]
\addcontentsline{toc}{chapter}{СПИСОК ИСПОЛЬЗОВАННЫХ ИСТОЧНИКОВ}

\bibliographystyle{utf8gost705u}  % стилевой файл для оформления по ГОСТу       % имя библиографической базы (bib-файла) 

\pagebreak
\end{document}